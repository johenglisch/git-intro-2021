\documentclass[12pt]{beamer}
\mode<presentation>

\usepackage[utf8]{inputenc}
\usepackage[T1]{fontenc}

\usepackage[british]{babel}

\usepackage{lmodern}

\title{Introduction to Git and Github}
\author{Johannes Englisch}
\date{27\,Aug 2021}


\setbeamertemplate{background}{%
  \includegraphics[width=\paperwidth,height=\paperheight]{images/bg-slide.jpg}%
}

\definecolor{shh-turquoise}{HTML}{08817F}
\definecolor{shh-link}{HTML}{7F0881}
\definecolor{shh-light-turquoise}{HTML}{A5DDD4}
\definecolor{shh-grey}{HTML}{D8D8D8}

\setbeamercolor{title}{fg=white}
\setbeamercolor{author}{fg=shh-grey}
\setbeamercolor{date}{fg=shh-grey}

%\setbeamercolor{titlelike}{fg=shh-turquoise}
\setbeamercolor{structure}{fg=shh-turquoise}
\setbeamercolor{section title}{fg=white}

\setbeamercolor{alerted text}{fg=shh-turquoise}
\setbeamercolor{navigation symbols}{fg=shh-turquoise}
\setbeamercolor{navigation symbols dimmed}{fg=shh-light-turquoise}

\AtBeginSection[]{%
  {%
    \setbeamertemplate{background}{%
      \includegraphics[width=\paperwidth,height=\paperheight]{images/bg-caption.jpg}%
    }%
    \begin{frame}
      \frametitle{}
      {\Huge{}\usebeamercolor[fg]{title}\insertsection{}}
    \end{frame}
  }%
}

\hypersetup{
  colorlinks,%
  linkcolor=,%
  urlcolor=shh-link,%
}


\begin{document}

{%
  \setbeamertemplate{background}{%
    \includegraphics[width=\paperwidth,height=\paperheight]{images/title.jpg}%
  }%
  \setbeamercolor{navigation symbols}{fg=shh-grey}%
  \setbeamercolor{navigation symbols dimmed}{fg=shh-light-turquoise}%
  \begin{frame}
    \titlepage%
  \end{frame}%
}

\section{What is Git?}

\begin{frame}
  \frametitle{Why do~I use git?}

  \begin{itemize}
    \item I~don't like to lose work
    \item I~like to have my work in a~place where people can find it
    \item I~like it when people work together without getting in each other's
      way
    \item I~like to be able to make mistakes
  \end{itemize}
\end{frame}

\begin{frame}
  \frametitle{Short fact sheet}

  \begin{columns}
    \begin{column}{.75\textwidth}
      \begin{itemize}
        \item Git is a~\alert{Distributed Version Control System}.
        \item Created in 2005 by \alert{Linus Torvalds}\\%
          (the creator of Linux).
        \item Probably the most popular VCS out there
      \end{itemize}
    \end{column}

    \begin{column}{.25\textwidth}
      {\tiny{}%
        \includegraphics[width=\textwidth]{images/linus-torvalds.jpg}\\%
        Photo: Krd, \href{https://creativecommons.org/licenses/by-sa/3.0}{CC BY-SA 3.0}, via Wikimedia Commons%
      }
    \end{column}
  \end{columns}
\end{frame}

\begin{frame}
  \frametitle{Version control}

  A~\alert{Version Control System} (VCS) keeps track of different versions of
  a project.
\end{frame}

\begin{frame}
  \frametitle{Two meanings of `version'}

  \begin{block}{Meaning 1}
    `Version' as in the state of a~project at a~given point in time.\\
    $\rightarrow$ i.\,e.\ `old version' vs.\ `new version'
    % TODO Illustration: Evolution of Firefox?
  \end{block}
\end{frame}

\begin{frame}
  \frametitle{Two meanings of `version' (ctd.)}

  \begin{block}{Meaning 2}
    `Version' as in different editions of the same project, existing next to
    each other.\\
    $\rightarrow$ e.\,g.\ `home version' vs.\ `enterprise version'
    % TODO Illustration: Firefox Rapid Release vs Firefox ESR?
  \end{block}
\end{frame}

\begin{frame}
  \frametitle{Two meanings of `version' (ctd.)}

  Version Control Systems do both!
  % TODO Illustration: Git tree
\end{frame}

\begin{frame}
  \frametitle{What can VCS's be used for?}

  \begin{itemize}
    \item Programming
    \item Configuration files
    \item Writing an article together in \LaTeX, Markdown, etc.
    \item \alert{Reproducible and citable research data}\\
      (we like that around here)
    \item Personal todo lists, notes, etc.
    \item These slides \texttt{;)}
  \end{itemize}
\end{frame}

\begin{frame}
  \frametitle{Know your limits!}

  \begin{block}{What are VCS's good at?}
    Text-based data
    (plain-text files, program code, XML/HTML, \LaTeX, CSV tables, etc.)
  \end{block}

  \begin{block}{What are (most) VCS's bad at?}
    Binary data
    (images, audio files, pdfs, zip archives, etc.)
  \end{block}
\end{frame}

\begin{frame}
  \frametitle{Centralised vs.\ Distributed VCS}

  \begin{block}{Centralised Version Control (e.\,g.\ CVS, subversion)}
    \begin{itemize}
      \item Central repository on some server somewhere
      \item People download the current state of the project to their computers
      \item People upload any changes they made back onto the server
    \end{itemize}
    % TODO Illustration: centralised VCS
  \end{block}
\end{frame}

\begin{frame}
  \frametitle{Centralised vs.\ Distributed VCS (ctd.)}

  \begin{block}{Distributed Version Control}
    \begin{itemize}
      \item Everybody has a~complete copy of the project (and all its history)
        on their hard drives
      \item Changes are pushed and pulled between the different repositories
    \end{itemize}
    % TODO Illustration: distributed VCS
  \end{block}
\end{frame}

\begin{frame}
  \frametitle{Centralised vs.\ Distributed VCS (ctd.)}

  \begin{block}{Note}
    In reality, people put a~separate copy of the project on a~server somewhere
    and pull/push their changes through that.
    \begin{itemize}
      \item Some project spin up their own git servers, e.\,g.:
        \begin{itemize}
          \item \href{https://git.kernel.org/}{Linux kernel$^{\hookrightarrow}$}
          \item \href{https://savannah.gnu.org/}{GNU operating system$^{\hookrightarrow}$}
        \end{itemize}
      \item Others use third-party hosting sites:
        \begin{itemize}
          \item \href{https://github.com}{Microsoft's Github$^{\hookrightarrow}$}
            (most popular)
          \item \href{https://bitbucket.org}{Atlassian's Bitbucket$^{\hookrightarrow}$}
          \item \href{https://about.gitlab.com}{Gitlab$^{\hookrightarrow}$}
        \end{itemize}
    \end{itemize}
  \end{block}
\end{frame}

\begin{frame}
  \frametitle{Know your limits! (ctd.)}

  \begin{block}{Advantages of Distributed Version Control}
    \begin{itemize}
      \item No need for a~constant internet connection
      \item Quick and comfy workflow (local branches are the best!)
      \item The massive redundancy reduces the chance of data loss
    \end{itemize}
  \end{block}

  \begin{block}{Drawbacks}
    \begin{itemize}
      \item Workflow of pulling/merging can be a~bit finicky at times.
      \item Copies get out of sync with each other more easily, increasing the
        number of merge conflicts.
      \item Copying around the entire history wastes disk space and bandwidth,
        and also just takes longer.
    \end{itemize}
  \end{block}
\end{frame}

% \begin{frame}
%   \frametitle{Quick terminology dump}
%
%   \begin{itemize}
%     \item \alert{Repository} (\emph{repo} or \emph{repos} for short):
%       \begin{itemize}
%         \item General tech speak:
%           online storage space
%         \item Git speak:
%           a~copy of a~version-controlled project folder
%       \end{itemize}
%     \item \alert{Cloning}:
%       downloading a~copy of an entire repo
%     \item \alert{Commit}:
%       a~single recorded change in a~project
%     \item \alert{Pulling}:
%       syncing commits \emph{from} a~remote git repo
%     \item \alert{Pushing}:
%       syncing commits \emph{to} a~remote git repo
%     \item \alert{Pull Request} (PR):
%       asking someone (kindly) to pull commits from your git repo into theirs
%   \end{itemize}
% \end{frame}

\section{Exploring a git repo}

\begin{frame}[fragile]
  \frametitle{Cloning an existing repo}

  {\footnotesize{}%
    \begin{verbatim}
$ git clone https://github.com/dictionaria/kalamang kalamang
Cloning into 'kalamang'...
remote: Enumerating objects: 155, done.
remote: Counting objects: 100% (155/155), done.
remote: Compressing objects: 100% (81/81), done.
remote: Total 155 (delta 81), reused 139 (delta 68), pack-reused 0
Receiving objects: 100% (155/155), 927.47 KiB | 7.73 MiB/s, done.
Resolving deltas: 100% (81/81), done.
    \end{verbatim}%
  }
\end{frame}

\begin{frame}[fragile]
  \frametitle{Status report!}

  {\footnotesize{}%
    \begin{verbatim}
$ cd kalamang
$ git status
On branch main
Your branch is up-to-date with 'origin/main'.

nothing to commit, working tree clean
    \end{verbatim}%
  }
\end{frame}

\begin{frame}[fragile]
  \frametitle{History lesson}

  {\footnotesize{}%
    \begin{verbatim}
$ git log
commit 5f28ae268e88d70e2f5f9ca2497a7e8fed257611 (HEAD -> main, tag: v1.1, origin/main, origin/HEAD)
Author: Robert Forkel <[...]@[...]>
Date:   Fri Apr 9 13:27:29 2021 +0200

    re-compiled against Glottolog 4.3

commit 3d98441885adb1037afcbde9478d8ae4c0330567
Author: Johannes Englisch <englisch@shh.mpg.de>
Date:   Wed Apr 7 14:41:57 2021 +0200

    regen with current pydictionaria version
[...]
    \end{verbatim}%
  }
\end{frame}

\begin{frame}[fragile]
  \frametitle{Time to time travel}

  {\footnotesize{}%
    \begin{verbatim}
$ git checkout <commit/branch/tag>
    \end{verbatim}
  }
\end{frame}

\section{Your very own git repo}

\begin{frame}[fragile]
  \frametitle{Creating a~new git repo}

  {\footnotesize{}%
    \begin{verbatim}
$ cd <folder>
$ git init
Initialised empty Git repository in <folder>/.git
    \end{verbatim}%
  }
\end{frame}

\begin{frame}
  \frametitle{Changing a~project is a~two-step process}

  \begin{enumerate}
    \item\alert{Add} all the changes you want to make to the Staging Area
    \item\alert{Commit} to your changes (and attach a~reasonably Commit Message)
  \end{enumerate}
  % TODO Illustration: commit process
\end{frame}

% add + status + commit + status
% gitignore
% add + commit again (checkout, reset)
% adding a tag?

\section{Connecting to the world wide web}

% register ssh key on github?
% make slide repo
% setup remote + pull
% push

% clone a separate version as proof

% add + commit
% push

% fetch + merge

\end{document}
